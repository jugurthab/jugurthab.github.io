\section{Available equipements}
Debugging Linux is a challenging task which requires a good preparation. In this section We present a global overview of some of the equipement used during the internship.
\subsection{Hardware platforms}


\subsubsection{Beagle Bone Black Wireless} 
The evolution of beaglebone black which adds wireless support (WIFI, Bluetooth) and fast linux boot (see \textbf{Figure \ref{Beaglebone black wireless}}). 
		\begin{figure}[H]
			\centering
        	\includegraphics[scale=0.25]{img/mean/beaglebone-black-wireless.png}
        	\caption{Beaglebone black wireless}
        	\label{Beaglebone black wireless}
    	\end{figure}
\begin{center}
\textbf{\color{red}Hardware specifications : } 
A datasheet is available at {\color{blue}\url{https://www.alliedelec.com/m/d/5505861ee370de1c82065dcc7bc77b0c.PDF}}.
\end{center}

\subsubsection{Raspberry PI 3 B+}
The lastest version as this time of writing with enhanced processor and ethernet speed (\textbf{Figure \ref{Raspberry PI 3}}).
		\begin{figure}[H]
			\centering
        	\includegraphics[scale=0.32]{img/mean/rpi3.png}
        	\caption{Raspberry PI 3}
        	\label{Raspberry PI 3}
    	\end{figure}
    	
\begin{center}	
\textbf{\color{red}Hardware specifications : } 
A datasheet is available at {\color{blue}\url{https://static.raspberrypi.org/files/product-briefs/Raspberry-Pi-Model-Bplus-Product-Brief.pdf}}.
\end{center}


\subsubsection{stm32f407 Board : } Used to build high performance applications oriented for audio processing (see \textbf{Figure \ref{stm32f407 Board}}).  
		\begin{figure}[H]
			\centering
        	\includegraphics[scale=0.08]{img/mean/stm-32.jpg}
        	\caption{stm32f407 Board}
        	\label{stm32f407 Board}
    	\end{figure}
    	
\textbf{\color{red}Specifications are available at :} can be found at {\color{blue}\url{https://www.st.com/content/ccc/resource/technical/document/user_manual/70/fe/4a/3f/e7/e1/4f/7d/DM00039084.pdf/files/DM00039084.pdf/jcr:content/translations/en.DM00039084.pdf}}    	


\subsubsection{AT32UC3C-EK Board : }
A developement kit for Atmel AVR microcontrolers (see \textbf{Figure \ref{AT32UC3C Board}}).  
		\begin{figure}[H]
			\centering
        	\includegraphics[scale=0.40]{img/mean/avr32.jpeg}
        	\caption{AT32UC3C Board}
        	\label{AT32UC3C Board}
    	\end{figure}
    	
    	\textbf{\color{red}Hardware spefications :}  available at {\color{blue}\url{http://www.farnell.com/datasheets/1511964.pdf}}

\subsubsection{ARM-USB-TINY-H JTAG Adapter : }  OpenOCD debugging interface adapter that uses the FTDI protocol (see \textbf{Figure \ref{ARM-USB-TINY-H}}).
		\begin{figure}[H]
			\centering
        	\includegraphics[scale=0.25]{img/mean/arm-usb-tiny-h.jpg}
        	\caption{ARM-USB-TINY-H}
        	\label{ARM-USB-TINY-H}
    	\end{figure}
Usage is described at : {\color{blue}\url{https://www.olimex.com/Products/ARM/JTAG/_resources/ARM-USB-TINY_and_TINY_H_manual.pdf}}\\

\textbf{\color{orange}Note : } OpenOCD supports multiple adapter protocols (ftdi, cmsis-dap, amt\_jtagaccel, remote\_bitbang, ..., etc). We can check the complet list at : {\color{blue}\url{http://openocd.org/doc/html/Debug-Adapter-Configuration.html#Debug-Adapter-Configuration}}. 
\subsection{Software}


\subsubsection{Pycharm IDE}
Pycharm is a python IDE, which makes developement fast. Examples of projects developed made in Python is an \textbf{OpenOCD wrapper} utility \og OESdebug \fg at : {\color{blue}\url{https://github.com/jugurthab/Linux_kernel_debug/tree/master/DebugSoftware/OpenOCD-wrapper}}.
\subsubsection{Eclipse C/C++ IDE}
Code examples were written maily in C, Eclipse C/C++ was helpful. Code are hosted on github at : {\color{blue}\url{https://github.com/jugurthab/Linux_kernel_debug/tree/master/debug-examples}}.

\subsubsection{OpenOCD}
Open source software allowing Hardware debugging, sources at maintained at : {\color{blue}\url{https://sourceforge.net/projects/openocd/files/openocd/}}.
