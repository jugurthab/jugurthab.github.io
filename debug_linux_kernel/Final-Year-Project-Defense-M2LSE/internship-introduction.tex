\section{Introduction}
{\Large Embedded devices are increasingly popular, devices are becoming smaller, smarter, interactive striving for better user experience.\\

Such a success was made possible since those tiny devices rely on UNIX-like operating systems (\textbf{Linux is the dominant}). 

\begin{itemize}
	\item[$\bullet$] \textbf{Open Source : } Sources are available and maintained by a large community, the latest stable version is available at {\color{blue}\url{https://www.kernel.org/}}\footnote{The lastest version (not stable) is available at Linus github : {\color{blue}sfdfd}}.
	
	\item[$\bullet$] \textbf{Not specific to vendor : } Linux is not propriatary operating system. We can point that major big companies are collaborators in it's developement.
	
	\item[$\bullet$] \textbf{Architecture support : } Linux supports many architectures (x86, arm, mips, ..., etc).
	
	\item[$\bullet$] \textbf{Low developement cost : } Linux is free. 
\end{itemize}

\vspace{10px}

However, such powerful operating systems are complex. Inconsistencies and logic flow errors can raise at any time (As the rule says : \og {\color{red}More code, more error prone} \fg),
We need mechanisms that can scale efficiently to track issues and bugs during developement and maintenance. A variety of tools have been adopted (some are even built-in) that help developers to write more stable and efficient applications.\\

\vspace{10px}

More can be said, as Linux is a multitasking and multiuser system. Every piece of code is checked for permissions. It
does even distinguish between two distinct spaces : \textbf{userspace} and \textbf{kernel}. each has it's own operating privileges (\textbf{kernel does have all the privileges}) so they must be debugged differently.

}